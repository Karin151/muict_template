\documentclass[12pt]{beamer}
\usetheme{mahidol}

\usepackage{amsmath}
\usepackage{amsfonts}
\usepackage{amssymb}
\usepackage{graphicx}

\usepackage{fontspec}
\usepackage{xunicode}
\usepackage{xltxtra}
\setsansfont{Cordia New}
%\setsansfont{DB Helvethaica X}  % another font suggest by Mahidol
% \setmainfont{Cordia New}       % is not effective on beamer
%\usepackage{setspace}           % for set line space

\usepackage{graphicx}
\usepackage{subcaption}

\usepackage{url}
\renewcommand{\UrlFont}{\ttfamily\tiny}

%color from https://en.wikibooks.org/wiki/LaTeX/Source_Code_Listings
\definecolor{mygreen}{rgb}{0,0.6,0}
\definecolor{mygray}{rgb}{0.7,0.7,0.7}
\definecolor{mymauve}{rgb}{0.58,0,0.82}

\usepackage{listingsutf8}
%\usepackage{listings}

\lstset{ %
    backgroundcolor=\color{black}, % choose the background color
    basicstyle={\color{mygray}\scriptsize\ttfamily},        % the size of the fonts that are used for the code
    columns=flexible,
    breakatwhitespace=false,       % sets if automatic breaks should only happen at whitespace
    breaklines=true,               % sets automatic line breaking
    captionpos=b,                  % sets the caption-position to bottom
    commentstyle=\color{mygreen},  % comment style
    deletekeywords={...},          % if you want to delete keywords from the given language
    escapeinside={(*}{*)},         % if you want to add LaTeX within your code
    extendedchars=true,            % lets you use non-ASCII characters; for 8-bits encodings only, does not work with UTF-8
    frame=single,	               % adds a frame around the code, (single, none)
    keepspaces=true,               % keeps spaces in text, useful for keeping indentation of code (possibly needs columns=flexible)
    keywordstyle=\color{mygray},   % keyword style
    language=bash,                 % the language of the code
    morekeywords={*,...},          % if you want to add more keywords to the set
    numbers=none,                  % where to put the line-numbers; possible values are (none, left, right)
    numbersep=10pt,                % how far the line-numbers are from the code
    numberstyle={\scriptsize\color{mygray}\ttfamily}, % the style that is used for the line-numbers
    rulecolor=\color{black},       % if not set, the frame-color may be changed on line-breaks within not-black text (e.g. comments (green here))
    showspaces=false,              % show spaces everywhere adding particular underscores; it overrides 'showstringspaces'
    showstringspaces=false,        % underline spaces within strings only
    showtabs=false,                % show tabs within strings adding particular underscores
    stepnumber=1,                  % the step between two line-numbers. If it's 1, each line will be numbered
    stringstyle=\color{mymauve},   % string literal style
    tabsize=2,	                   % sets default tabsize to 2 spaces
    title=\lstname                 % show the filename of files included with \lstinputlisting; also try caption instead of title
}

\usepackage[
    backend=bibtex,
    bibstyle=ieee,
    natbib=true,
    style=numeric-comp,
    hyperref=false,
    isbn=false,
    doi=false]{biblatex}

\addbibresource{reference.bib}

\renewcommand*{\bibfont}{\scriptsize}
\setbeamertemplate{bibliography item}[text]
\setbeamerfont{footnote}{size=\scriptsize} 

\author{Phitchayaphong Tantikul}

\title{This is My Topic}
%\subtitle{}

\institute{Ph.D. student, Faculty of ICT,\\
    Mahidol University\\\vspace{.5em}
    Advisor: ..... \vspace{.5em}}

\date{\today}

\setbeamercovered{transparent}
\setbeamertemplate{navigation symbols}{}

% ----- start document -----
\begin{document}
\maketitle

% ----- table of contents -----
\begin{frame}{Overview}
    \tableofcontents
\end{frame}

% ***** Section 1 *****
\section{Introduction}

% ----- Page 1 -----
\begin{frame}{Statistics for Web Attacks}
    Something...
    \footfullcite{SymantecThreatReport2016}
\end{frame}

% ----- Page 2 -----
\begin{frame}{Web Application Attacks}
    \small
    \textbf{\large OWASP Top-10 (v. 2013)}\\
    \textbf{Most Critical Web Application Security Risks} \footfullcite{OWASP-Top10}
    \begin{itemize}\setlength\itemsep{-0.5em}
        \item A1-Injection
        \item A2-Broken Authentication and Session Management
        \item A3-Cross-Site Scripting (XSS)
        \item A4-Insecure Direct Object References
        \item A5-Security Misconfiguration
        \item A6-Sensitive Data Exposure
        \item A7-Missing Function Level Access Control
        \item A8-Cross-Site Request Forgery (CSRF)
        \item A9-Using Components with Known Vulnerabilities
        \item A10-Unvalidated Redirects and Forwards
    \end{itemize}
\end{frame}

% ----- Page 3 ----- ([fragile] for page that contains source code)
\begin{frame}[fragile]{Web Application Attacks: SQL Injection}
    \footnotesize
\begin{lstlisting}[language=php]
$sql = "SELECT * FROM users 
        WHERE username='admin' AND password='".$_POST['password']."'";
mysql_query($sql);
\end{lstlisting}

    What if \lstinline|$_POST['password']| contains string ``\textbf{\color{red}' or '1'='1}''
    
    \medskip Then, \lstinline|$sql| will be
    
    ``SELECT * FROM users WHERE username='admin' AND password='\textbf{\color{red}' or '1'='1}' ''
    
    \medskip It is equivalent to
    
    "SELECT * FROM users WHERE \textbf{\color{red} true}"
    
    which returns back everything in the table.
    
    \medskip \begin{center}
        \normalsize
        \textbf{This let attacker eventually gains access\\
            to view or manipulate other data on the server}
    \end{center}
\end{frame}

% ***** Section 2 *****
\section{Security Countermeasures}
\begin{frame}{Security Countermeasures}
    Something...
    
    \bigskip
    \fullcite{OWASP-SQL-Prevention}
\end{frame}

% ...

% ***** Last Section *****
% ----- Summary -----
\begin{frame}{Summary}
    Summary:
    \begin{itemize}
        \item abc
        \item def
    \end{itemize}
\end{frame}

% ----- Q & A -----
\begin{frame}{Q \& A}
    \centering\large
    Q\&A and Suggestions
\end{frame}

\begin{frame}{References}
    \begin{block}{}
        \printbibliography[heading=none]
    \end{block}
\end{frame}

\end{document}